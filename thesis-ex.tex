%%
%% This is file `thesis-ex.tex',
%% generated with the docstrip utility.
%%
%% The original source files were:
%%
%% uiucthesis2014.dtx  (with options: `example')
%% 
\def\fileversion{v2.25b} \def\filedate{2014/05/02}
%% Package and Class "uiucthesis2014" for use with LaTeX2e.
\documentclass[edeposit,fullpage]{uiucthesis2014}



\usepackage[utf8]{inputenc}

\usepackage{graphicx}
\usepackage{amsmath}
\usepackage{amsthm}
\usepackage[colorlinks,allcolors=blue]{hyperref}
\usepackage{array, tabularx, caption, boldline}
\usepackage{makecell}
\usepackage{cellspace}
\usepackage[citestyle=numeric-comp]{biblatex}

\addbibresource{thesisbib.bib}
\addbibresource{astrid.bib}
\addbibresource{astrid-missing.bib}
\addbibresource{siesta.bib}
\addbibresource{svdquest.bib}
\addbibresource{hgt.bib}
\addbibresource{fastrfs.bib}



\theoremstyle{definition}
\newtheorem{thm}{Theorem}
\newtheorem{lemma}[thm]{Lemma}
\newtheorem{theorem}[thm]{Theorem}
\newtheorem{definition}[thm]{Definition}
\newtheorem{cor}[thm]{Corollary}
\newtheorem{rem}[thm]{Remark}
\newtheorem{remark}[thm]{Remark}
\newtheorem{conj}[thm]{Conjecture}
\newtheorem{observation}[thm]{Observation}

\pdfstringdefDisableCommands{%
  \let\enspace\empty  % this causes the warning for \kern
  \let\noindent\empty % this causes the warning for\indent
  \let\hskip\empty
}


\begin{document}

\title{Large Scale Phylogenomic Estimation}
\author{Pranjal Vachaspati}
\department{Computer Science}
%\schools{B.A., University of Columbia, 1981\\
%         A.M., University of Illinois at Urbana-Champaign, 1986}
\phdthesis
\advisor{Tandy Warnow}
\degreeyear{2019}
\committee{Professor Prof Uno, Chair\\Professor Prof Dos, Director of Research\\Assistant Professor Prof Tres\\Adjunct Professor Prof Quatro}
\maketitle

\frontmatter

%% Create an abstract that can also be used for the ProQuest abstract.
%% Note that ProQuest truncates their abstracts at 350 words.
\begin{abstract}
\end{abstract}

%% Create a dedication in italics with no heading, centered vertically
%% on the page.
\begin{dedication}
\end{dedication}

%% Create an Acknowledgements page, many departments require you to
%% include funding support in this.
\chapter*{Acknowledgments}


%% The thesis format requires the Table of Contents to come
%% before any other major sections, all of these sections after
%% the Table of Contents must be listed therein (i.e., use \chapter,
%% not \chapter*).  Common sections to have between the Table of
%% Contents and the main text are:
%%
%% List of Tables
%% List of Figures
%% List Symbols and/or Abbreviations
%% etc.

\tableofcontents
\listoftables
\listoffigures

%% Create a List of Abbreviations. The left column
%% is 1 inch wide and left-justified
\chapter{List of Abbreviations}

\begin{symbollist*}
\item[AD] Average distance between true gene trees and true species tree (a measure of ILS)
\item[CA-ML] Concatenated Maximum-Likelihood
\item[GTR] Generalized Time Reversible
\item[HGT] Horizontal Gene Transfer
\item[ILS] Incomplete Lineage Sorting
\item[ML] Maximum-Likelihood
\item[RF distance] Robinson-Foulds distance
\end{symbollist*}

%% Create a List of Symbols. The left column
%% is 0.7 inch wide and centered
%\chapter{List of Symbols}

%\begin{symbollist}[0.7in]
%\item[$\tau$] Time taken to drink one cup of coffee.
%\item[$\mu$g] Micrograms (of caffeine, generally).
%\item[$n$] Number of taxa
%\end{symbollist}

\mainmatter
\part{Introduction}

\chapter{Phylogenomic estimation}


The goal of phylogenomic estimation is to estimate evolutionary trees
from genomic data. An evolutionary tree is a representation of the
evolutionary history of the organisms being studied. Finding accurate
evolutionary trees is an interesting scientific problem in itself, and
these trees are also key components of a number of downstream
biological analyses.

As genomic sequencing costs continue to fall dramatically \cite{},
cutting-edge phylogenomic analyses increases in two dimensions: in the
number of organisms (taxa) studied; and in the amount of genetic
information considered per taxon. Upcoming analyses, including the
next phase of the Avian Phylogenomics Project \cite{}, the 10,000 plat
genome project \cite{}, the Genome 10k project \cite{}, the i5k
arthropod genome project \cite{}, and others, will analyze whole
genomes of thousands or tens of thousands of taxa.

The scale of this data presents unique computational challenges, as
many existing methods were designed to run on datasets with tens or
hundreds of taxa. In addition to running efficiently on large
datasets, new methods must also estimate accurate trees on datasets
generated under a range of biological conditions that complicate
phylogenomic analyses, including incomplete lineage sorting and
horizontal gene transfer.

\section{Organization of this work}

Chapter \ref{chapter:background} provides much of the background
necessary to understand this research into phylogenomic methods,
including the relatively minimal amount of biology needed to describe
the relevant mathematical models of evolution.

\subsection*{Part \ref{part:supertree}}

Part \ref{part:supertree} focuses on supertree estimation. The goal of
supertree estimation is to combine small trees on subsets of a larger
taxon set into a single tree on the entire taxon set. Supertree
estimation is commonly used to combine results of smaller analyses (as
in \cite{seabirds,cpl,thpl,marsupials,placental}).

A supertree method can also be used as a component of a
divide-and-conquer technique for species tree estimation. These
techniques \cite{} divide a large dataset into many small, overlapping
datasets, and run a species tree estimation method on each
subset. Then, a supertree method is used to combine trees on the
subsets into a tree on the entire taxon set. In this way, methods that
may be too slow or memory intensive to run on a large dataset directly
may still be used to analyze that dataset.

We describe two methods for supertree estimation: FastRFS and ASTRID

Chapter \ref{chapter:fastrfs} introduces FastRFS
\cite{fastrfs}. Designed as a supertree method, FastRFS uses a
constrained optimization technique to exactly solve its NP-hard
optimization criterion within a constrained search space.

Chapter \ref{chapter:siesta} describes SIESTA, an improvement to the
constrained optimization algorithm used in FastRFS that allows it to
consider multiple optimal solutions to its optimization criterion and
return consensus trees of those solutions.

Chapter \ref{chapter:astrid-missing} describes modifications to ASTRID
(which is discussed in more detail in the species-tree context in
Chapter \ref{chapter:astrid}) that allow it to be used effectively for
very large species tree analyses.

\subsection*{Part \ref{part:speciestree}}

Part \ref{part:speciestree} discusses methods for species tree
estimation. These methods take evolutionary trees on individual genes
(gene trees) as input, which may differ from one another due to
various biological effects, and return a species tree, which
represents the actual evolutionary history of the organisms.

Chapter \ref{chapter:astrid} describes ASTRID, a distance matrix based
method for species tree estimation. ASTRID provides highly accurate
species trees, and is capable of analyzing extremely large datasets in
a short amount of time.

Chapter \ref{chapter:svdquest} introduces SVDquest, which is an
implementation of the SVDquartets method that uses a constrained
optimization technique to exactly optimize the SVDquartets
optimization criterion within a constrained search space. This allows
accurate species trees to be estimated directly from alignments in a
statistically consistent manner.

Chapter \ref{chapter:hgt} analyzes the behavior of phylogenetic
estimation methods on datasets with incomplete lineage sorting (ILS)
as well as horizontal gene transfer (HGT). 


\chapter{Background}
\label{chapter:background}
\section{Phylogenetic trees}

The use of a phylogenetic tree to describe evolutionary relationships
was popularized by Darwin, who used a diagram of a tree as the sole
illustration in \emph{On the Origin of Species}
\cite{darwin1859origin}. While the methods used to estimate these
trees have changed drastically, the basic structure and meaning of
these trees is more or less the same.

Phylogenetic trees have \emph{nodes} (\emph{leaves} and \emph{internal
  nodes}), and \emph{edges}.

The leaves of a phylogenetic tree represent extant taxa that have been
sampled for the analysis, referred to as the taxon set.

Internal nodes represent speciation events. An internal node also
represents a species that is the most recent common ancestor of all
the descendants of that node, and in some cases (for example, when
fossil data is used), may correspond to a known species.

Edges represent the evolution of a species without any speciation
events that led to multiple extant species in the dataset (that is to
say, speciation events may have occurred along an edge, but only one
of those species survived to the present day and was included in the
analysis). Edges may have a length, which represents the amount of
time between speciation events.


Each edge also corresponds to a split, or bipartition, in the taxon
set. The descendants of an internal node $n$ are a clade, and the
parent edge of that node separates that clade from the rest of the
taxa, which do not have $n$ as an ancestor.


%\subsection{Rooted and unrooted trees}

Phylogenetic trees may be rooted or unrooted. In many models of
evolution, the root is not identifiable; in other words, every rooting
of the same unrooted tree will produce the same distribution of site
patterns on the leaves of the tree. Accurately identifying the root of
an unrooted tree can be challenging in some cases, especially for gene
trees that evolve under ILS or HGT (see Section
\ref{sec:heterogeneity}).
%TODO figure

Trees may be binary or multifurcating (non-binary). Each internal node
in a binary tree has degree $3$ (except for the root, which has degree
$2$), while internal nodes in multifurcating trees may have higher
degrees. Since speciation events are in reality binary,
multifurcations in trees represent an uncertainty as to the ordering
of two or more speciation events.
% TODO figure

\subsection{Distances between trees}

The distance between two trees that share a set of taxa can be
measured in a few different ways, the most common of which is the
Robinson-Foulds (RF) distance. Each edge in the trees corresponds to a
split of the taxon set, so a tree can be represented as a set of
bipartitions of the taxon set. The RF distance between two trees is
the percentage of bipartitions that are shared between the two trees;
in other words:

\begin{equation}
  \label{eq:rfdistance}
  RF(T_1, T_2) = \frac{|bp(T_1) \cap bp(T_2)|}{|bp(T_1 \cup bp(T_2)|}
\end{equation}
TODO CHECK

The RF distance is commonly used to measure the accuracy of an
estimation method with respect to a known true tree from a
simulation.

Other tree distance metrics are also used, including triplet and
quartet distances. A triplet is a rooted tree on three leaves, and a
quartet is an unrooted tree on four leaves. For three taxa, there are
three possible rooted triplet topologies, and for four taxa, there are
three possible unrooted quartet topologies. Triplets and quartets are
the smallest informative rooted and unrooted trees; there is only one
possible rooted tree on two taxa and one possible unrooted tree on
three taxa. The triplet or quartet distance between two trees is the
proportion of quartet topologies shared between the trees.

\section{Character sequences}

The input to phylogenetic estimation problems is often an alignment
matrix containing DNA, RNA, or amino acid sequences. Each row in the
alignment corresponds to a single taxon, and each column represents a
single character, which may take various character states (depending
on the type of the sequences; a DNA character can take the states
$\{A,C,T,G\}$), or $-$, representing a character insertion or deletion
(``indel'').

\section{A simple model of evolution}
\label{section:simple-model}


The generalized time-reversible (GTR) mathematical model of evolution
is a Markov process, where each character evolves independently, and a
character transitions from one state to another along an edge $e$
depending on the length of $e$ and a diagonally symmetrical \emph{rate
  matrix}. 

This model is useful because computing the relative likelihoods of
different trees given a sequence alignment is not difficult, and
maximum-likelihood tree estimators are an important phylogenetic tool.

\section{Gene tree heterogeneity}
\label{sec:heterogeneity}

However, evolution is in reality more complex than the GTR model
suggests. This is because different parts of the genome evolve in
different ways, with different evolutionary histories and evolutionary
trees. This can happen for a number of reasons, two of which are
horizontal gene transfer (HGT) and incomplete lineage sorting (ILS).

We refer to a portion of the genome that has a single evolutionary
history as a \emph{c-gene} or often as just a \emph{gene}. This is
different from the standard biological definition of a gene, (i.e. a
genetic sequence that codes for a particular protein). The
evolutionary history of a gene is captured in a \emph{gene tree}, and
differences between these gene trees are referred to as \emph{gene
  tree heterogeneity}.

\subsection{Horizontal gene transfer (HGT)}

% TODO figure
The easiest to understand cause of gene tree heterogeneity is
horizontal (or lateral) gene transfer (HGT or LGT). An HGT event
occurs when two organisms from different species exchange DNA. There
are many biological reasons for this. Bacteria, for example, emit DNA
in the form of short circular segments called plasmids, and other
bacteria can readily consume these plasmids and add them to their own
DNA. Bacterial evolutionary trees have high levels of HGT, and for
many analyses, it is more appropriate to think of bacterial evolution
as represented by a network rather than a tree. Eukaryotic organisms
also experience HGT, although much less frequently than bacteria. A
common example is hybridization, where two different species can mate
to produce fertile offspring. 


\subsection{Incomplete lineage sorting (ILS) and the coalescent model}

Incomplete lineage sorting (ILS) is a much more common cause of gene
tree heterogeneity among eukaryotes. This process is modeled by the
multispecies coalescent. 

\section{Phylogenomic estimation under the coalescent model}

Maximum likelihood estimation under the 

\subsection{Short loci}

%things outside of the scope of my thesis

\section{Techniques for phylogenomic estimation}

\subsection{Concatenated maximum-likelihood (CA-ML) analyses}
\subsection{Bayesian Markov chain Monte Carlo methods}
\subsection{Coalescent-aware methods}
\subsubsection{ASTRAL}
\subsubsection{MP-EST}
\subsubsection{NJst}

\section{Supertree estimation}
\subsection{Methods}
\subsubsection{MRL}
\subsection{Divide and conquer methods}

\part{Supertree Estimation}
\label{part:supertree}

\chapter{FastRFS}
\label{chapter:fastrfs}
\input{fastrfs.tex}

\chapter{Improving Dynamic Programming for Phylogenomic Estimation
  with SIESTA}
\label{chapter:siesta}
\input{siesta.tex}
\section{Supplementary data for SIESTA}
\input{siesta-supplement.tex}

\chapter{Supertree Estimation with ASTRID}
\label{chapter:astrid-missing}
\input{astrid-missing.tex}




\part{Species Tree Estimation}
\label{part:speciestree}

\chapter{Species Tree Estimation with ASTRID}
\label{chapter:astrid}
\input{astrid.tex}


\chapter{SVDquest}
\label{chapter:svdquest}
\input{svdquest.tex}

\chapter{Species Tree Estimation with ILS and HGT}
\label{chapter:hgt}
\input{hgt.tex}


\chapter{Conclusions}

\chapter{Bibliography}
\printbibliography

\backmatter


\end{document}
\endinput
%%
%% End of file `thesis-ex.tex'.
